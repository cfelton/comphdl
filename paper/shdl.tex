
\section{A New Language}

This is a start, which shall go into it's own paper, of a new language.
Name is needed, but the current acronym is SHDL.

SHDL shall be a simple, minimalistic language to generate synthesizable
hardware via an intermediate language, such as VHDL. SHDL shall
be simple to learn and it shall be clear which constructs are synthesizable.
Maybe all constructs are synthesizable and a simulation needs another
language to implement the test bench.

One suggestion is that we implement the new DSL as an embedded DSL in
Modelyze. The important thing is, however, that this DSL is
\emph{exclusively embedded}, meaning that it does not derive more
functionality from the host language (Modelyze in this case) than
necessary or wanted. For instance, our new HDL should perhaps require
all translations to VHDL to terminate and only allow static
typing. We may use the term \emph{fencing} for describing
how the embedded DSL is separated from the host language.


\subsection{Short Notes}

\martin{I like the x.next notion from MyHDL for register values as it shows
that the value will change on the next clock tick and is not yet available.
However, MyHDL uses this for combinational signals as well, which is not
so good.}


\subsection{Concepts}
We should make explicit and clear fundamental concepts

\begin{itemize}
\item Component abstraction
\item Component instances
\item Wiring between components
\item Registers
\item Logic
\item Parameterization
\end{itemize}


\subsection{Simple circuits with logic only}

\shdl\lstset{}
\begin{lstlisting}[]
comp Xor2()
  in i1, i2 : bool
  out o1 : bool
  o1 = i1 xor i2
end

comp Not1()
  in i1 : bool
  out o1 : bool
  o1 = !i1
end

comp SimpleCircuit()
  in  a : bool[3] 
  out even : bool = unit3.o1

  def unit1 = Xor2()          
  unit1.i1 = a[0] 
     
  def unit2 = Xor2()
  unit2.i1 = unit1.o1
  unit2.i1 = a[2]          
 
  def unit3 = Not1()         
  unit3.i1 = unit2.o1
end
\end{lstlisting}

We use def to define a local name. In the above case is it used for naming component instances.

\shdl\lstset{}
\begin{lstlisting}[]
type R1 = {
  a:int
  b:int
}

type R2 = {
  a:int
  b:int
}

type R3 = {
  a:int
  b:int
}

comp A()
  out o2:R2
  out o3:R3
end

comp B()
  in i4:R4
  in i3:R3
  out o1:R1
end

comp C
  in i1:R1
  in i2:R2
  out o4:R4
end

comp Circuit
  def a = A()
  def b = B()
  def c = C()
  b.i3 = a.o3
  b.i4 = c.04
  c.i1 = b.o1
  c.i2 = a.o2
end


comp Counter
  in clear:bool
  reg cnt:int4 = a 
  def a = if !clear then cnt + 1 else 0
  out v:int4 = cnt
end
\end{lstlisting}


Use term “single assignment”
