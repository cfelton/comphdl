
\documentclass[10pt, conference, compsocconf]{IEEEtran}


%\usepackage{pslatex} % -- times instead of computer modern

\usepackage{listings}
\usepackage{courier}

\usepackage{url}
\usepackage{booktabs}
\usepackage{graphicx}
%\usepackage{caption}
%\usepackage{subcaption}
\newcommand{\code}[1]{{\small{\textsf{#1}}}}
\newcommand{\codefoot}[1]{{\textsf{#1}}}
\usepackage{cite}
\usepackage{amsmath,amsthm}
\usepackage[usenames,dvipsnames]{xcolor}

\newcommand{\todo}[1]{{\emph{TODO: #1}}}

\usepackage{listings}
%\usepackage{subfigure}

\newcommand{\comment}[3]{\paragraph*{\textbf{#1}}{\color{#3}#2}}


\newcommand{\martin}[1]{\comment{Martin}{#1}{Blue}}
\newcommand{\david}[1]{\comment{David}{#1}{Blue}}


\begin{document}

% MKL listings
\newcommand{\shdl} {
  \lstset{keywords={comp,end,def,in,out,if,then,else,true,false,
      int,real,bool,string,int1,int2,int3,int4,int5,int6,int7,int8,
      int16,int32,type}, 
    morecomment=[l]{//},
    morecomment=[s]{/*}{*/}, morestring=[b]{"},
    basicstyle=\ttfamily\small, showstringspaces=false,
    keywordstyle=\bfseries, mathescape=true, }}


\title{Working title: Compare HDLs or\\
Are New Hardware Description Languages Available for FPAG
Development and Teaching Digital Design?}

\author{Martin Schoeberl\\
Department of Applied Mathematics and Computer Science, 
Technical University of Denmark\\masca@imm.dtu.dk
  \vspace{1ex}\\ 
David Broman\\
University of California, Berkeley and Link{\"o}ping University\\
broman@eecs.berkeley.edu
}



\maketitle \thispagestyle{empty}

\begin{abstract}
Compare different HDLs.
\end{abstract}

\section{General}

Do some exploration of user base and available source code on all languages.

How many \emph{background} knowledge on programming language do we need for the new HDL?

Can it be a language to teach 1st and 2nd semester EE students to learn digital
electronics? Even without programming background? We now struggle with VHDL.

Languages to look at:

http://tomahawkins.org/ two languages at the bottom

MyHDL
Chisel
Lava
Gezel
JHDL
Confluence
HDCamel


\section{Introduction}

\subsection{Assessment Criteria}


In synchronous design the two basic building blocks are combinational
(asynchronous) logic and flip-flop based registers. Larger on-chip
memories are not built out of registers, but are considered another building
block. This building blocks are also reflected in current FPGAs: lookup tables
(LUT) are used to implement combinational functions, dedicated flip-flops
the registers, on-chip memories support larger storage requirements,
and dedicated hard macros (with MAC operations) support efficient
implementation of DSP algorithms.

This results in following questions:
\begin{itemize}
\item Is it clear what is combinational logic and what are registers?
\item Can FPGA block RAMs be instantiated? Can ROMs be used?
\item Can the DSP blocks be used?
\end{itemize}

\todo{Write some text around the questions as motivation for them}

Further questions to be answered:

\begin{itemize}
\item Is simulation in the HDL possible?
\item Is there a clear line between the synthsizable  and not synthsizable parts of the language?
\item Can it be used in a first semester digital electronics course?
\item Quality of generated code: hardware resources and maximum clock frequency
\item Mixed language (with VHDL and/or Verilog) usage?
\item Size of the user group and support
\item Active development
\end{itemize}

Missing and hard to assess: productivity, easy to maintain.

\section{Getting Started}

This is a first tour through the languages.

\subsection{MyHDL}

Starting with MyHDL was a pleasure. Having Python already installed, the installation
of MyHDL was a matter of minutes. Having been exposed to Python for about one
day before, the first examples from the MyHDL manual~\cite{myhdl:2010} where easy as well.
However, jumping ahead and trying to generate some real VHDL code and programming
an FPGA with the HW version of \emph{Hello World} took some hours struggling
with cryptic error messages. The author is used to statically typed languages and
compile checks, where Pythons dynamic typing and interpreting execution model
reveals program and even typing errors only at runtime needs some adaption of
the authors programming mindset and habits.

The development approach in MyHDL is to test and verify the hardware within Python first
before generating VHDL or Verilog. Therefore, basic Python language knowledge is
needed for MyHDL. In fact MyHDL is less a language in its own, but a Python package.

\todo{Maybe this goes into a Discussion section at the end}

The question is if a dynamic typed language, where a variable can change the type
at runtime, the best base for HW synthesis is. In the resulting Verilog/VHDL code and
in the hardware the type of a signal or a register is statically fixed. Type errors in
Python are only detected at runtime by extensive testing. In a statically typed
language the compiler finds all type errors at compile time.

\todo{Comment: Python is easy going with dynamic typing and test benches run.
However, when converting to VHDL or Verilog, types (i.e., lengths of bit vectors)
need to be specified. As a result after high level simulation more work is needed
to have synthesizable code.}





\begin{itemize}
\item Very simple to download install
\item Python needs to be learned, some VHDL/Verilog knowledge is still needed
\end{itemize}

Installed on Python 2.7. Does it need a 2.x Python or would it run on 3.x as well?

How large is the user base on MyHDL? Any open-source projects?

see \url{http://thread.gmane.org/gmane.comp.python.myhdl/2701}

\subsubsection{Remarks}

Why does the manual show code for a simple mix that is not convertible to VHDL?

This is not the way we would like the process for synchronous logic:

\begin{verbatim}
FOO_HDL: process (clk, reset) is
begin
    if (reset = '1') then
        sig <= '0';
    elsif rising_edge(clk)or falling_edge(reset) then
        sig <= to_std_logic((not to_boolean(sig)));
    end if;
end process FOO_HDL;
\end{verbatim}

The MyHDL manual introduces too early, too many non-synthesizable constructs.
They might be good for test benches, but that should be stated more explicit.

Positive is conversion to VHDL and Verilog.

\subsection{Lava}

Lava is a HDL based on Haskell~\cite{Lava:1998}. The first version described
is focused on verification and does not yet support description of sequential
hardware. \todo{Assume that this is all gone and there are some more papers.}

\martin{Looks like this Lava thing is tight with Xilinx. VHDL code generation
is also mentioned.}

\section{Language Assessments}

In this section we will assess which languages fulfill the criteria proposed in
the introduction and answer the questions.

\subsection{Is it clear what is combinational logic and what are registers?}

\todo{Check Verilog}

In VHDL there is no clear distinction between combinational logic, flip-flop
registers,  and memories in the language. Registers (and memories) are
inferred from \emph{code patterns}. Furthermore, small errors can result
in unintended latches generated in processes that are intended to be combinational.

MyHDL has no distinct types for register of memory. However, it uses annotations
(so called decorators in Python) to mark a function representing
combinational logic with \code{@always\_comb} and a function where the
output is stored in registers with \code{@always(clk.posedge)}. We consider
this as an enhancement related to VHDL/Verilog \todo{Check Verilog}.

\subsection{Documentation and Tutorials}

The MyHDL distribution contains a mental~\cite{myhdl:2010} that includes
a tutroal.

\subsection{Active Community}

It is important that a new language has a active developer group that is
able to fix errors and incorporate changes when new (FPGA) technology
becomes available. Furthermore, an active forum for the language users
is beneficial for getting started.

MyHDL is developed and maintained by Jan Decaluwe with the help of
Christopher Felton.  

\section{Notes (from iPad/Dropbox)}

MyHDL

Having min and max values that cannot be implemented so in HW is strange. I would prefer just signed/unsigned with a power of 2 range.

MyHDL is not a new language, but a python package. Python is the language. If a HW description is embedded in a full blown general purpose language with a lot of libraries, how easy is it to see the boundaries of what is synthesizable hardware, what is test benches, unit tests, and even generator code?

There is no indication is parameters are input or output signals.

Testing looks convenient with all the Python support.

Are signal records/structures possible?

A dynamic (scripting) language depends very much on unit tests. Even typos are only
discovered when the code is executed. Static typed languages (such as Java
and Scala) can catch quite a lot of bugs and typos during compilation.

General

Is tri state supported?

Shall I look into C based HDLs?

SystemC?

SystemVerilog

\section{Further Notes and Reading Material}

MyHDL news group archive: \url{http://blog.gmane.org/gmane.comp.python.myhdl}




\section{A New Language}

This is a start, which shall go into it's own paper, of a new language.
Name is needed, but the current acronym is SHDL.

SHDL shall be a simple, minimalistic language to generate synthesizable
hardware via an intermediate language, such as VHDL. SHDL shall
be simple to learn and it shall be clear which constructs are synthesizable.
Maybe all constructs are synthesizable and a simulation needs another
language to implement the test bench.

One suggestion is that we implement the new DSL as an embedded DSL in
Modelyze. The important thing is, however, that this DSL is
\emph{exclusively embedded}, meaning that it does not derive more
functionality from the host language (Modelyze in this case) than
necessary or wanted. For instance, our new HDL should perhaps require
all translations to VHDL to terminate and only allow static
typing. We may use the term \emph{fencing} for describing
how the embedded DSL is separated from the host language.


\subsection{Short Notes}

\martin{I like the x.next notion from MyHDL for register values as it shows
that the value will change on the next clock tick and is not yet available.
However, MyHDL uses this for combinational signals as well, which is not
so good.}

Most languages handle as primary target circuit simulation and
synthesis as after thought. This often leads to a language, where it
is not clear which constructs can be synthesized and which not.
However, we shall approach the circuit
design with \emph{Synthesize first} to have a clear notion of synthesizable
constructions in the language.


\subsection{Concepts}
We should make explicit and clear fundamental concepts

\begin{itemize}
\item Component abstraction
\item Component instances
\item Wiring between components
\item Registers
\item Logic
\item Parameterization
\end{itemize}


\subsection{Simple circuits with logic only}

\shdl\lstset{}
\begin{lstlisting}[]
comp Xor2()
  in i1, i2 : bool
  out o1 : bool
  o1 = i1 xor i2
end

comp Not1()
  in i1 : bool
  out o1 : bool
  o1 = !i1
end

comp SimpleCircuit()
  in  a : bool[3] 
  out even : bool = unit3.o1

  def unit1 = Xor2()          
  unit1.i1 = a[0] 
     
  def unit2 = Xor2()
  unit2.i1 = unit1.o1
  unit2.i1 = a[2]          
 
  def unit3 = Not1()         
  unit3.i1 = unit2.o1
end
\end{lstlisting}

We use def to define a local name. In the above case is it used for naming component instances.

\shdl\lstset{}
\begin{lstlisting}[]
type R1 = {
  a:int
  b:int
}

type R2 = {
  a:int
  b:int
}

type R3 = {
  a:int
  b:int
}

comp A()
  out o2:R2
  out o3:R3
end

comp B()
  in i4:R4
  in i3:R3
  out o1:R1
end

comp C
  in i1:R1
  in i2:R2
  out o4:R4
end

comp Circuit
  def a = A()
  def b = B()
  def c = C()
  b.i3 = a.o3
  b.i4 = c.04
  c.i1 = b.o1
  c.i2 = a.o2
end


comp Counter
  in clear:bool
  reg cnt:int4 = a 
  def a = if !clear then cnt + 1 else 0
  out v:int4 = cnt
end
\end{lstlisting}


Use term “single assignment”



\section{Conclusion}
\label{sec:conclusion}



\bibliographystyle{IEEEtran}
%\bibliographystyle{abbrv} % similar to IEEE without URLs
% pleas add bibs into other.bib as msbib.bib is 'generated'
\bibliography{msbib}

\end{document}

